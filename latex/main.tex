%%% Local Variables:
%%% mode: latex
%%% TeX-master: t
%%% End:

\documentclass[bachelor,nofonts]{thuthesis}
%\documentclass[master]{thuthesis}
%\documentclass[doctor]{thuthesis}
% \documentclass[%
%   bachelor|master|doctor|postdoctor, % mandatory option
%   winfonts|nofonts|adobefonts, % mandatory only for bachelor and Linuxer
%   secret,
%   openany|openright,
%   arialtoc,arialtitle]{thuthesis}
% 当使用 XeLaTeX 编译时,本科生、Linux 用户需要加上 nofonts 选项;
% 当使用 PDFLaTeX 编译时,adobefonts 选项等效于 winfonts 选项(缺省选项)。

% 所有其它可能用到的包都统一放到这里了,可以根据自己的实际添加或者删除。
\usepackage{thutils}

% 你可以在这里修改配置文件中的定义,导言区可以使用中文。
% \def\myname{薛瑞尼}

\begin{document}

% 定义所有的eps文件在 figures 子目录下
\graphicspath{{figures/}}


%%% 封面部分
\frontmatter
\input{data/cover}
%\makecover
\makecover{thu-whole-logo.pdf}{thu-lib-logo.pdf}

% 目录
\tableofcontents

% 符号对照表
\input{data/denotation}


%%% 正文部分
\mainmatter
\include{data/chap01}
\include{data/chap02}


%%% 其它部分
\backmatter

% 本科生要这几个索引,研究生不要。选择性留下。
\makeatletter
\ifthu@bachelor
  % 插图索引
  \listoffigures
  % 表格索引
  \listoftables
  % 公式索引
  \listofequations
\fi
\makeatother


% 参考文献
\bibliographystyle{thubib}
\bibliography{ref/refs}


% 致谢
\include{data/ack}

% 附录
\begin{appendix}
\input{data/appendix01}
\end{appendix}

% 个人简历
\include{data/resume}
\end{document}
